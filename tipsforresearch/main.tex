\documentclass{article}

% Language setting
% Replace `english' with e.g. `spanish' to change the document language
\usepackage[english]{babel}

% Set page size and margins
% Replace `letterpaper' with `a4paper' for UK/EU standard size
\usepackage[a4paper,top=2cm,bottom=2cm,left=3cm,right=3cm,marginparwidth=1.75cm]{geometry}

% Useful packages
\usepackage{amsmath}
\usepackage{graphicx}
\usepackage[colorlinks=true, allcolors=blue]{hyperref}

\title{Notes on Deep Learning Research}
\author{Standford CS230: 8th Session}

\begin{document}
\maketitle

\begin{abstract}
A few tips on how to do research on Deep Learning and Machine Learning research.
\end{abstract}

\section{Introduction}
\begin{enumerate}
    \item Compile list of papers (research + articles)
    \item Skip/Skim around this list (decide which one is cool and go ahead and read it, then move on to the next one.
    \item The ones you don't understand perhaps there will be a pre-request paper.
    \item 5-20 papers in an area: good  understanding of the area to do research.
    \item 50-100 papers read in an area: very very good understanding of the subject.
\end{enumerate}

\section{Reading One Paper}
Bad way to read a paper: go from start to finish.
A couple papers a week is a good steady speed
\subsection{How To Read A Paper}
Take multiple passes to a paper:
1: Read the title/abstract/figures\\
2: Read the intro + conclusions + figures + skim the rest (skim or skip related work)\\
3: Read the paper but just skip/skim the math.\\
4: Read the whole thing but skip past the stuff that doesn't make sense.\\
5: If you want you can continue to read everything in depth (dont do this if you are trying to go through papers, its about time prioritization.)\\
What happens is that some papers have stuff that is very important down the line but is written as a sideline stuff.. e.g net5 paper by
Yann Lecun had foundations for CNNs but half of it was about something else that didn't see that much use later.\\
Now you have to ask yourself some questions after you've read it:\\
What did the authors try to accomplish?\\
What were the key elemnts of the approach?\\
What can you use yourself?\\
What other references do you want to follow?\\
You have to keep up with new papers but how?\\
- Twitter is pretty good\\
- ML sub reddit\\
- Conferences\\
- NIPS/ICML/ICLR <glance thru the titles see whats cool to you> \\
- Friends \\

\section{Math}
If you want to deeply understand something or some algorithm... re-derive from scratch after reading it (make sure you really really understand it... after 
placing the results aside , Rederive... its hot.. its a must almost)
WHY this? if you do this, then you will able to get better at deriving things... so later u will be able to generalize this and derive your own work. 
like how art students learn by copying the work of great artists in the art gallery
same is for code

\section{Code}
-lvl1: run open source-code (easy)
-lvl1: reimplement from scratch (this is where the learning happens)

Learn steadily over short bursts of ultimate focus  (like ur just reading a few papers all the time in the holidays)
e.g spaced repition v cramming, you will have more long term retention

\end{document}